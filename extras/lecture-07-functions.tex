\documentclass{tufte-handout}

\usepackage{xcolor}

% set hyperlink attributes
\hypersetup{colorlinks}

% set image attributes:
\usepackage{graphicx}
\graphicspath{ {images/} }

% indent subsections
\newenvironment{subs}
  {\adjustwidth{3em}{0pt}}
  {\endadjustwidth}


% ============================================================

% define the title
\title{SOC 4015/5050: Lecture 07 Functions}
\author{Christopher Prener, Ph.D.}
\date{Fall 2018}
% ============================================================
\begin{document}
% ============================================================
\maketitle % generates the title
% ============================================================

\vspace{5mm}
\section{Packages}
\begin{itemize}
\item \texttt{base}
\item \texttt{stats}
\end{itemize}

\vspace{3mm}
\section{Writing a Function}
\noindent {\color{red}\texttt{\textit{functionName}}}\texttt{ <- base::}{\color{red}\texttt{function}}\texttt{(\textit{param1}, \textit{param2})\{}\\
\texttt{\textit{\# function body}}\\
\noindent \texttt{\}}

\vspace{5mm}
\section{Calculating Absolute Value}
\texttt{stats::}{\color{red}\texttt{abs}}\texttt{(\textit{x})}

\vspace{5mm}
\section{t Distribution}
For the \textit{t} distribution, let: \\
\noindent $t =$ score\\
\noindent $df =$ degrees of freedom\\

\begin{subs}

\vspace{3mm}
\subsection{Basic Function}
\noindent \texttt{stats::}{\color{red}\texttt{pt}}\texttt{(q = \textit{t}, df = \textit{df})}\\

\vspace{3mm}
\subsection{Full Equation}
\noindent \texttt{2*stats::}{\color{red}\texttt{pt}}\texttt{(q = -stats::{\color{red} abs}(\textit{t}), df = \textit{df})}

\end{subs}

\newpage
\section{The Full Probability Under t Function}
\begin{verbatim}
#' Two-tailed Probabilities Under the t Distribution
#' 
#' @description This function calculates the probability of observing a t score
#'     at least as extreme as the given t value.
#'     
#' @param t A given t score
#' @param n The sample size associated with t
#'
#' @return A probability value     
#'      
probt <- function(t, n){
 
  # calculate the degrees of freedom given n
  df <- n-1
  
  # calculate the p value
  out <- 2*pt(q = -abs(t), df = df)
  
  # return output
  return(out)
   
}
\end{verbatim}


% ============================================================
\end{document}